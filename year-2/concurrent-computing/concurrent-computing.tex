%%=====================================================================================
%%
%%       Filename:  concurrent-computing.tex
%%
%%    Description:  Formal notes of concurrent computing
%%
%%        Version:  1.0
%%        Created:  02/10/19
%%       Revision:  none
%%
%%         Author:  Josh Felmeden (), nk18044@bristol.ac.uk
%%   Organization:  
%%      Copyright:  Copyright (c) 2019, Josh Felmeden
%%
%%          Notes:  
%%
%%=====================================================================================

% Preamble {{{
\documentclass[11pt,a4paper,titlepage,dvipsnames,cmyk]{scrartcl}
\usepackage[english]{babel}
\typearea{12}
% }}}

% Set indentation and line skip for paragraph {{{
\setlength{\parindent}{0em}
\setlength{\parskip}{1em}
\usepackage[margin=2cm]{geometry}
\addtolength{\textheight}{-1in}
\setlength{\headsep}{.5in}
% }}}

\usepackage{hhline} 
\usepackage{mathtools} 
\usepackage[T1]{fontenc}

% Headers setup {{{
\usepackage{fancyhdr}
\pagestyle{fancy}
\lhead{Concurrent Computing: The guide}
\rhead{Josh Felmeden}
\usepackage{hyperref} 
% }}}

% Listings {{{
\usepackage[]{listings}
\usepackage[]{xcolor} 
\definecolor{RoyalBlue}{cmyk}{1, 0.50, 0, 0}
\lstset{language=[Visual]Basic,
    keywordstyle=\color{RoyalBlue},
    basicstyle=\small\ttfamily,
    commentstyle=\ttfamily\itshape\color{gray},
    stringstyle=\ttfamily,
    showstringspaces=false,
    breaklines=true,
    frameround=ffff,
    frame=single,
    rulecolor=\color{black}
}
% }}}

% Listings {{{
\definecolor{lstgrey}{rgb}{0.05,0.05,0.05}
\usepackage{listings}
\makeatletter
\lstset{language=[Visual]Basic,
    backgroundcolor=\color{lstgrey},
    frame=single,
    xleftmargin=0.7cm,
    frame=tlbr, framesep=0.2cm, framerule=0pt,
    basicstyle=\lst@ifdisplaystyle\color{white}\footnotesize\ttfamily\else\color{black}\footnotesize\ttfamily\fi,
    captionpos=b,
    tabsize=2,
    keywordstyle=\color{Magenta}\bfseries,
    identifierstyle=\color{Cyan},
    stringstyle=\color{Yellow},
    commentstyle=\color{Gray}\itshape
}
\makeatother
\renewcommand{\familydefault}{\sfdefault}
% }}}


% Other packages {{{
\usepackage{needspace}
\usepackage{tcolorbox}
\usepackage{soul}
\usepackage{babel,dejavu,helvet} 
\usepackage{amsmath} 
\usepackage{booktabs} 
\usepackage{tcolorbox} 
\usepackage[symbol]{footmisc} 
\renewcommand{\thefootnote}{\fnsymbol{footnote}}
\renewcommand{\familydefault}{\sfdefault}
% }}}

% Title {{{
\title{Concurrent Computing: The guide}
\author{Josh Felmeden}
% }}}

\begin{document}

\maketitle
\tableofcontents

\newpage
\section{Sequential Processes}%
\label{sec:sequential-processes}

\section{Fast Fourier Transform}%
\label{sec:fft}

\subsection{Polynomials}%
\label{sub:Polynomials}
A \textbf{degree} $n-1$ polynomial in $x$ can be seen as a function:

\begin{align*}
    A(x) = \sum^{n-1}_{i=0}a_i \cdot x^i
\end{align*}

Any integer that's bigger than the degree of a polynomial is a
\textit{degree bound} of said polynomial. The polynomial $A$ is:

\begin{align*}
    a_0 \cdot x^0 + a_1 \cdot x^1 + a_2 \cdot x^2 \cdots + a_{n-1}x^{n-1}
\end{align*}

The values $a_i$ are the \textit{coefficients}, the degree is $n-1$ and
$n$ is a degree bound. We're able to express any integer as some kind of
polynomial by setting $x$ to some base, say for decimal numbers:

\begin{align*}
    A = \sum^{n-1}_{i=0} a_i \cdot 10^i
\end{align*}

The variable $x$ just allows us to evaluate the polynomial at a point. A
really fast way to evaluate the polynomial is to use \textbf{Horner's
Rule}.

\begin{tcolorbox} [space to upper,
        collower=white,
        title={Horner's Rule},
        nobeforeafter,
        halign lower=flush right, ]
Instead of computing all the terms individually, we do:

\begin{align*}
    A(3) = a_0 + 3 \cdot (a_1 + 3\cdot (a_2 + \cdots + 3\cdot (a_{n-1})))
\end{align*}

This method requires $O(n)$ operations

\end{tcolorbox}





\end{document}
