% Preamble {{{
\documentclass[11pt,a4paper,titlepage,dvipsnames,cmyk]{scrartcl}
\usepackage[english]{babel}
\typearea{12}
% }}}

% Set indentation and line skip for paragraph {{{
\setlength{\parindent}{0em}
\setlength{\parskip}{1em}
\usepackage[margin=2cm]{geometry}
\addtolength{\textheight}{-1in}
\setlength{\headsep}{.5in}
% }}}

\usepackage{hhline}
\usepackage[table]{xcolor}
\usepackage{mathtools}
\usepackage[T1]{fontenc}

% Headers setup {{{
\usepackage{fancyhdr}
\pagestyle{fancy}
\lhead{Internet Economics and Financial Technology}
\rhead{Josh Felmeden}
\usepackage{hyperref}
% }}}

% Listings {{{
\usepackage[]{listings}
\lstset
{
    breaklines=true,
    tabsize=3,
    showstringspaces=false
}

\definecolor{lstgrey}{rgb}{0.05,0.05,0.05}
\usepackage{listings}
\makeatletter
\lstset{language=[Visual]Basic,
backgroundcolor=\color{lstgrey},
frame=single,
xleftmargin=0.7cm,
frame=tlbr, framesep=0.2cm, framerule=0pt,
basicstyle=\lst@ifdisplaystyle\color{white}\footnotesize\ttfamily\else\color{black}\footnotesize\ttfamily\fi,
captionpos=b,
tabsize=2,
keywordstyle=\color{Magenta}\bfseries,
identifierstyle=\color{Cyan},
stringstyle=\color{Yellow},
commentstyle=\color{Gray}\itshape
}
\makeatother
\renewcommand{\familydefault}{\sfdefault}
\newcommand{\specialcell}[2][c]{%
\begin{tabular}[#1]{@{}c@{}}#2\end{tabular}}
% }}}

% Other packages {{{
\usepackage{graphicx}
\graphicspath{ {./pics/} }
\usepackage{needspace}
\usepackage{tcolorbox}
\usepackage{soul}
\usepackage{babel,dejavu,helvet}
\usepackage{amsmath}
\usepackage{booktabs}
\usepackage{tcolorbox}
\usepackage[symbol]{footmisc}
\renewcommand{\thefootnote}{\fnsymbol{footnote}}
\renewcommand{\familydefault}{\sfdefault}
\usepackage{enumitem}
\setlist{nolistsep}
% }}}

% tcolorbox {{{
\newtcolorbox{blue}[3][] {
colframe = #2!25,
colback = #2!10,
#1,
}

\newtcolorbox{titlebox}[3][] {
colframe = #2!25,
colback = #2!10,
coltitle = #2!20!black,
title = {#3},
fonttitle=\bfseries
#1,
}
% }}}

% Title {{{
\title{Internet Economics and Financial Technology}
\author{Josh Felmeden}
% }}}

\begin{document}
\maketitle
\tableofcontents
\newpage
\section{The Big Picture}
Based on past events, it is possible that we have reached the climax of the IT revolution. The migration to cloud technology is mirrored by the migration of electricity from home generators to power stations. From here, there have only been marginal improvements in this field, and this could be the same in the tech sector.

Similarly, this was seen in the explosion of the `.com boom' seen later. This rapid growth was due to people investing lots of money in the technology; a mania. Again, this has been seen before in history with the development of canals.

Another bubble popping phenomena was the collapse of the American housing market, where the stock market collapsed. Humans are really good at messing up financially, and also inventing tech that can revolutionise the world.

Essentially, what I'm getting at is that in the financial sector, trading is mostly done by computers. Obviously, these computer traders have no common sense, which worries a lot of the big companies, and investigations have been done into this.

\subsection{Expanding the big picture}
This section is the only section where we will discuss the history of finance, so bear with me here. Looking at the last 250 years of technology `surges', we can possible gain insights about the current surge.

\begin{enumerate}
\item Industrial revolution (1770-1873)
\item Steam and railways (1829-1873)
\item Steel, electricity (1875-1918)
\item Oil, Car, Mass production (1908-1974)
\item IT and Telecoms (1971-??)
\end{enumerate}

If the IT surge ends at the mean duration, then we can expect the surge to end in 2024. This end does not mean the end of the world, just that the time to make great fortunes has passed.

So, computer science is no longer \textit{just} about computer science. Because of this surge, there is great interplay between this field and others, such as finance.

\section{Big Money}
\subsection{Positive Feedback}
Successful companies have won out for a whole host of factors, but history tends to focus on the winners. \textit{Positive feedback} and \textit{network externalities} can help to elevate a company or product to success, even if it has superior rivals.

\subsection{The Long Tail}
The long tail was named and popularised by Chris Anderson. In the old days, Retailers could make money on high-volume, low-margin goods or low-volume, high-margin goods. This is because shops had shelf volume that, if filled up by low-volume, low-margin goods, would be making a loss on these products, since physical space costs money. Thanks to the advent of online markets, `shelf' space costs almost nothing, meaning that shops can now stock the low-volume, low-margin goods.

In examples such as the music industry, these less popular goods still have returns that would be beneficial for the vendor. In fact, no matter how far down the popularity graph you go, there is still money to be made. This tail is called the \textit{long tail}.

\subsubsection{Power-Law distribution of popularity}
The power law appears a weirdly large amount in terms of popularity, which is of the form $P = cR^{-v}$. This makes the graph dip very sharply early on, but level off; never reaching zero.

As it turns out, the biggest money could be made in the smallest scales (if they are done large scale). The way to do it is:
\begin{enumerate}
\item Make everything available
\item Reduce prices
\item `Help me find it'
\end{enumerate}

\subsubsection{Criticism of The Long Tail}
Does the long-tail effect really exist? One observation is that the web actually \textit{magnifies} the importance of `blockbuster' hits. However, Anderson disagrees with this, stating that it is all about where you define the head and tail of the power-law curve and whether you use absolute values (since the criticism compares percentages).

Another criticism stated that music sales exhibited a log-normal distribution rather than a power-law curve. They reported that 80\% of the music tracks they monitored sold NO copies at all over a one-year period. This was disputed poorly by Anderson.

\subsubsection{Support for The Long Tail}
The Long Tail has been proven to have grown longer over time. Niche books now account for around 37\% of Amazon's sales. Additionally, a longer but also fatter tail has been observed on consumer software downloading patterns. However, this is only for software, which would behave differently than the entertainment sector.

\subsection{Disruptive Technology}
A technology company may have some form of technology and a projected performance. The company will also have an idea of the improvement required by the mainstream market. As long as the technology offered improves faster than the mainstream market's demands, the company is going to do fine. However, these are just predictions, and therefore this prediction must be monitored.

Some time into the future, a new technology appears with a much smaller performance and market than yours. However, it may have another benefit (smaller/cheaper/lighter), meaning that other companies invest in this. Because our company is already doing well, it is not a threat and we are not interested in it. The research team also conclude that the technology will not improve as fast as ours, therefore, our technology will always outperform the new one.

This prediction is true for a long time. At a certain point, the needs of the mainstream market are met by the new technology. Despite our technology being far above the needs of the consumer, the market is taken away from us, because our technology is far too advanced, and the new technology has other benefits that the mainstream market find more attractive, causing our company to be obsolete.

\subsubsection{The Innovator's Dilemma}
There are two different types of innovations:
\begin{itemize}
    \item \textbf{Sustaining Innovations}: Incremental improvements on existing products or services that are attractive to existing customers and business models. Eventually, you offer more than the customer wants
    \item \textbf{Disruptive Innovations}: perform perform less well than existing products, perhaps being lower quality or less sophisticated, but they are also simpler, cheaper, or more user friendly.
\end{itemize}

These disruptive innovations can cause strong incumbent companies to fall or falter, not due to weaknesses in these companies, but because they do the right think short-term; leaving the lower end of the market to others.

The \textit{dilemma} is the choice between doing what made the company a success in the first place, or investing in a lower quality prospect, meaning that sometimes successful incumbents need to invest in the `wrong thing'.

Very often, this is the sequence of events:
\begin{itemize}
    \item The disruptive technology is developed by the incumbent company
    \item The existing customers are unimpressed
    \item Therefore, the company stops developing the disruptive technology and instead concentrates on sustaining innovations
    \item New companies form (sometimes from disgruntled ex-employees of said company) and develop a new market for the disruptive technology
    \item As the disruptive technology matures and improves, it moves `up the chain'
    \item The company then realises that there is significant demand, and attempts to enter as a latecomer. It fails to do this due to the lead built already by these new companies.
\end{itemize}

\section{Internet Economics}
\subsection{Micro Economics}
Micronomics studies the behaviour of individuals and businesses and how decisions are made based on the allocation of limited resources. This income can then be used to make or provide other resources, giving us a cycle. Consumers have a finite amount of resources, so the price of the goods needs to be well placed so the consumers will buy. But also, consumers have a finite amount of resources, so the supply also needs to be carefully placed. These interactions can become quite complicated quite quickly.

Supply and demand is at the heart of this. On the one hand, we have the amount demanded by the consumers, and on the other, the amount supplied by the producers. The \textbf{equilibrium} of these two values is when they are the same.

Excess demand means that consumers are fighting over the finite goods, meaning that the price rises; perhaps above the equilibrium. Some consumers will be unlikely to buy at this inflated price, meaning that the equilibrium returns.

Excess supply causes prices to fall, and some producers are unlikely to want to continue creating these goods, returning the equilibrium.

Every individual has a different amount of a certain good that they demand. If we aggregate these demands, we get a \textit{demand curve}. The law of diminishing marginal benefit means that the demand will have a downward slope as we go up in quantity. The \textit{gradient} of the curve tells us the \textbf{price elasticity} of demand. If the gradient is small (i.e. small elasticity) a small change in price results in a large change in quantity demanded. Conversely, \textit{inelastic} demand means that a large change in price results in a small change in quantity demanded.

How quantity demanded changes in relation to a price:
\begin{itemize}
    \item Horizontal demand: \textbf{perfect elasticity} (a change in quantity has no effect on price)
    \item Vertical demand: \textbf{perfect inelasticity} (fixed quantity demanded, regardless of price
    \item 45-degrees: \textbf{unit elasticity} (percent change in price equals percent change in demand)
\end{itemize}

\subsection{Production costs}
Running a business has associated costs:
\begin{itemize}
    \item \textbf{Fixed costs}: a company must incur fixed costs in order to operate. They are fixed during a short time period.
    \item \textbf{Variable costs}: these tend to rise with increased production
    \item \textbf{Semi-variable}: labour can be considered a variable cost, because if paying overtime hours, or if you decide to hire more people to cope with increased workload.
\end{itemize}

We can graph the costs against quantity produced. \textbf{Economies of scale} are increased profit when the company is first scaling. After this, we see \textbf{diminishing marginal returns}. The marginal cost curve tends to be upwards sloping, and firms need to cover their marginal cost of each additional unit produced. Therefore, the minimum price they will supply their product is the MC. They also need to cover their AVC. The supply curve for a firm is the curve that appears above the production costs.

As with elastic price, there is also elastic supply. \textbf{Elastic supply} can accomodate a large chang ein supply for relatively small additional costs, while \textbf{inelastic supply} means an increase in suply requires high additional costs.

\subsubsection{Competitive markets}
A \textbf{competitive market} creates efficient allocation of resources at equilibrium where quantity generated matches the quantity demanded. If there is an excess of demand, the price will increase, and vice versa.

Supply and demand ccurves show the price-quantity relationship. A change in another factor can shift the demand/supply. This shift can be represented by a change in either line causing a new equilibrium.

\subsubsection{Monopolistic Markets}
Sometimes, as single firm grows to take over an entire market. This firm would be called a \textit{monopolist}. In the real world, complete monopoly is rare, and so if a firm owns more than a quarter of the market, it is considered a monopoly. In a competitive market, firms ar eprice takers. Each firm supplies a very small proportion of the market so no matter how much they produce, there will only be demand at a price. Conversely, for a monopolist, they are the only provider, so the demand curve is the same as the market demand curve. They are able to set the price to whatever they like, since they have no competition.

\subsection{Economics of the Internet}
The internet has disrupted the business landscape:
\begin{itemize}
    \item \textbf{Combinatorial innovation} --- components can be combined and recombined to create new products or services --- has accelerated due to the internet.
    \begin{itemize}
        \item 1800s, standardised gears revolutionised manufacturing revolutionised manufacturing
        \item Internet revolution --- innovations are rapidly distributed globally, many innovations are open-source and standardisation is easierand more necessary.
    \end{itemize}
    \item Economic laws have not fundamentally changed, but the caharacteristics of online business activities can result in different markets.
\end{itemize}

Online economic activities consist of:
\begin{itemize}
    \item Digital goods
    \item Information goods
    \item Online purchasing of physical goods
    \item Online provision of services
\end{itemize}

Digital differences between digital and physical goods:
\begin{enumerate}
\item Digital goods tend to be costly to produce, but cheap to reproduce. Fixed costs are high, but variable costs are low
\item Most of the production costs are \textit{sunk} (cannot be recovered) unlike a factory, for example, which can be resold after
\item There are no capacity constraints limiting the number of times something can be reproduced.
\item Digital goods are often \textit{experience goods} (customer downt know the value without using them)
\item Searching online is easy, so search costs for the customer is very low
\item \textit{Positive network externalities} are often strong --- the value to you increases as more people use it.
\end{enumerate}

\subsubsection{Consumers}
If the production and consumption of a good or service affects the third party who was not involved, then they are a \textit{externality}. A \textbf{negative externality} imposes costs on the others (such as pollution). \textbf{Positive externality} has benefits (such as spending on education).

\textbf{Network externalities} is a kind of externality that occurs when the act of buying a product confers indirect cost or value on all those who already own the same product or service. A \textbf{Negative network externality} imposes costs on others who also own the product (such as car purchase increasing road congestion). \textbf{Positive network externality} provides benefit to others who also own the product or service (such as telephone that makes you available for others to call). The term \textit{network effect} is often used in a positive way.

The network efect is a very powerful effect. The demand curve of this is initially upward sloping, meaning that the initial marginal cost is very large. This is because the network is growing, and so is of more benefit to the consumer. After a while, this tapers off, and the demand curve ends up looking hyperbolic. The midway point in the upward curve is the \textit{critical mass}. This is the quantity of users that must be reached for the network to be viable.

\textbf{Switching} from one product from one product or service to another can have additional cost. When this cost becomes very high, a user is said to be \textit{locked in}. In the digital domain:
\begin{itemize}
    \item The consumer would have to retrain to learn another service or software
    \item The consumer would have to convince their friends/peers to switch too
    \item The sotftware needs to be set up again
    \item There will be reduced quality of service in the short term because of the information that is stored by one company that will not be transferred to the other.
\end{itemize}

\subsubsection{Producers}
For the competitive markets:
\begin{itemize}
    \item Short term: supply curve is MC above AVC (average variable cost)
    \item Long term: supply curve is MC above ATC (average total cost)
\end{itemize}

If two companies are competeing with identical digital products as their varaible costs are near zero, copetition will drive the price down to near-zero. A new company won't want to rist competing with an existing company because the set up costs are high compared with the possible profits. The risk is further enhanced by the newtwork effect as any new product must reach the critical mass or it will fail. Additionally, switching costs and lock in mean htat users are less likely to swap to the new product. Therefore, for commodity digital ogods, there is a tendency towards monopoly. To succeed, companies must therefore focus on product differentiation or competing for emerging future monopolies.

For digital products with a strong network effect and high switching costs, we expect to see monopolies occuring.

Producers greatly gain from having a monopoly and therefore will try to retain this. \textit{Proprietary formats} are a powerful means of enforcing monopoly, such as the \textit{.doc} format of Word. OpenOffice reverse engineered this, reducing lock-in, but switching costs meant that it was not widely uptaken. As a result, Word is still dominant, but OpenOffice is making inroads.

\textbf{Standards} help to stop this. Adopting industry-wider standards allows a user network to be shared between providers. This seems detrimental, but the network value is greatly increased. There are multiple ways a standard can be introduced:
\begin{enumerate}
\item Standards have a \textit{leader}, that sets the standard for the industry
\item A \textit{standard war}, where two or more producers compete to dominate
\item A \textit{standards negotiation}, where two or more producers negotiate a standard collectively
\end{enumerate}

\subsection{Pricing Strategies}
There are three degrees of \textbf{price discrimination}
\begin{itemize}
    \item \textbf{First-degree}: Perfect price discrimination. The business charges the maximum possible price for each unit sold. It is hard to know everyones individual spending habits, so is not often used anymore
    \item \textbf{Second-degree}: When a company charges a different quantities consumed, such as quantity disounts on bulk purchases, or versioning (premium, standard, free tiers). This means buyers self-select and reveal their type via their selection. Much easier to employ this in the digital world.
    \item \textbf{Third-degree}: Grouping a product by a certain group, such as cinema tickets with students, adults, children, etc. Buyers cannot choose their group.
\end{itemize}

For price discriminatino to take place, a seller must satisfy three conditions:
\begin{itemize}
    \item Distinguish between customers
    \item Have enough market power
    \item Resale must be impractival, costly, or forbidden
\end{itemize}

Price discrimination can also be controversial for a variety of reasons, and sometimes against the law.

Bundling is another form of versioning and is the practice of selling several goods together for a single price, such as Microsoft Office. It's used to sell to customers who would otherwise not buy. Can help to keep a monopoly.

\section{Auctions}
\subsection{Auction Theory}
There are four common auction types:
\begin{itemize}
    \item Open Ascending-Price (English)
    \item Open Descending-price (Dutch)
    \item First-price sealed bid
    \item Second-price sealed bid (Vickrey)
\end{itemize}

Auctions are not new, and have been used since the era of the Babylonians. Auctions are used because the \textit{seller is unsure} about the values that bidders attach to the object being sold. If the seller knew bidder values, he could just offer the object to the bidder with the highest value $v$, or just below.

\subsubsection{Terminology}
\textbf{Private values}:
\begin{itemize}
    \item Each bidder knows the value of the object to themself at the time of the bidding
    \item No bidder knows with certainty the values attached by \textit{other} bidders, and knowledge of the other bidders' values have no effect
\end{itemize}

This assumption is most plausible when the value of the object to a bidder is derived from its consumption or use alone, such as a pizza or a bicycle.

\textbf{Interdependence} is when the object worth may be \textit{unknown} at the time of the auction to the bidder.
\begin{itemize}
    \item Bidder may have only an \textit{estimate} or some private signal that is correlated with the true value.
    \item Other bidders may possess \textit{information} that, if known, would affect the value that a particular bidder attaches to the object.
\end{itemize}

We call this specification \textbf{interdependent values} which are values that are unknown at the time of the auction and may be affected by the information available to other bidders.

This assumption is most plausible for situations in which the object being sold that can possible be resold after the auction, such as a Harry Potter first edition book.

\subsubsection{Equivalent Auctions}
\begin{itemize}
    \item \textbf{First-price sealed-bid auction}:
    \begin{itemize}
        \item The bidder's strategy maps private information to a bid
        \item No information about other bidders is available as the auction is sealed
        \item The bidder needs to consider what the bid, as they have no idea what other people might bid
    \end{itemize}
    \item \textbf{Dutch auction}:
    \begin{itemize}
        \item Conducted in the open, but offers no useful information to bidders
        \item The only information available is that some bidder has agreed to buy at the current price, but that causes the auction to end.
    \end{itemize}
\end{itemize}

These two are equivalent because bidding a certain amount in the FPSB auction is equivalent to offering to buy at that amount in a Dutch auction. Therefore, Dutch open descending price auction is strategically equivalent to the first-price sealed-bid auction.

\begin{tcolorbox}
\begin{center}
    \textbf{strategic Equivalence}: for every strategy in one game, a player has a strategy in another game that results in the same outcomes
\end{center}
\end{tcolorbox}

\begin{itemize}
    \item \textbf{English auction}
    \begin{itemize}
        \item Offers information when other bidders drop out. By observing this, it may be possible to infer something about their privately known information
        \item with \textit{private values}, however, this information is of no use
    \end{itemize}
    \item \textbf{English Auction Strategy}
    \begin{itemize}
        \item It cannot be optimal to:
        \begin{itemize}
            \item Stay in after the price, $p$ exceeds the value $v$ (incurring a loss)
            \item Drop out before the price reaches the value (losing out on potential gains)
        \end{itemize}
        \item Therefore, the optimal strategy is to bid up to the value $p = v$
    \end{itemize}
\end{itemize}

the strategy for a \textbf{Second-price sealed-bid auction} is also the same:
\begin{itemize}
    \item Bidder $B$ with private value $v$ bids a price $p$.
    \item The highest competing bid has price $c$
\end{itemize}

So, the optimal bidding strategy for this auction is:
\begin{itemize}
    \item If $B$ bids at $p = v$:
    \begin{itemize}
        \item $B$ wins if $v = c$ (making profit $v-c$) and does not win if $v < c$.
        \item If $v=c$, assume $B$ is indifferent between winning and losing
    \end{itemize}
    \item Alternatively, if $B$ bids $p$ lower than $v$ (i.e. $p<v$):
    \begin{itemize}
        \item If $v > p > c$, then $B$ still wins and profit is still $v-c$
        \item If $c > v > p$, then $B$ still loses, so also no change, BUT
        \item If $v > c > p$, the $B$ now loses, so makes less profit
        \item Therefore, bidding $p < v$ can never increase profit for $B$ but can decrease profit.
    \end{itemize}
\end{itemize}

We say that these auctions are \textbf{weakly equivalent} because the two auctions are not strategically equivalent, but the optimal strategies are the same if the values are private.

With interdependent values, the information available to others in the open auction is relevant to a bidder's evaluation of worth. Seeing some other bidder drop out early may make the bidder realise that his own estimate is too great. Therefore, if the values are interdependent, the two auctions may not be equivalent from the perspective of the bidders.

\begin{tcolorbox}
\begin{center}
    An auction is said to be \textit{incentive compatible} if it encourages bidders to bid their \textit{true value of the good}
\end{center}
\end{tcolorbox}

Both English and Vickrey are incentive compatible, since when values are private, the optimal strategy is to bid $p = v$.

However, Dutch and FPSB are not, since we want to bid lower than the true value; attempting to stop the auction just below the price of what other bidders value the good at.

Incentive compatible auctions stop game-playing between bidders.

\subsubsection{Revenue Equivalence Theorem (RET)}
\begin{itemize}
    \item In English/Vickrey auctions, bidders bid price $p = v$.
    \item In Dutch/FPSB, bidders price $p < v$.
    \item However, the expected revenue in a first-price auction is the same as the expected revenue in a second-price auction.
\end{itemize}

The revenue equivalence theorem states that if the values are private, and bidders are risk neutral, any standard auction will yield the \textbf{same expected revenue} to the seller.

The revenue equivalence theorem requires \textit{risk neutral} bidders, meaning that they have no emotional attachment to the bids and they only seek to maximise expected profits. However, in reality, bidders are normally \textit{risk averse}, meaning that they will bid higher, as they will buy \textit{insurance} against the possibility of losing. When we take risk aversion into account, we get the result that the expected equivalence in the first-price auction is greater than that in a second-price auction.

The RET also requires private values, but in reality, the values are often interdependent, since we often change our valuations based on valuations from others. When we take interdependence into account, ordinary ascending auctions are more profitable than standard (first-price) auctions.

\subsection{Auctions in Practice}
Auctions don't always perform in the way that we expect. Too much economic knowledge can sometimes be a dangerous thing. Essentially, if you can't prove the mathematics in theory, then it's not worth having the theory in the first place.

There are two key implicit assumptions, where all auctions are equally attractive and there is no collusion between bidders. However, attractiveness and robustness to collusion are important in practical auction design. Profitability depends on the \textit{number of bidders} who participate. Participating can be costly, so bidders will only enter if they feel they can win. In \textbf{ascending auctions}, a strong bidder can always top a weak bidder, so weak bidders are not likely to enter. On the other hand, in a first-price sealed bid, a weaker bidder might win, because a stronger bidder is trying to cop a deal. Unfortunately, because bidders can't observe each other, they are not able to influence the auction.

It is possible to signal other bidders and collude with them to ensure that you both get a deal for a really cheap price.

\subsection{Online Auctions}
There are a lot of platforms for auctions online; allowing for auctions to be run quickly and cheaply. The auction rules are algorithmic, so there is no need for a physical auctioneer. Open auctions become easier and cheaper for both the seller and the buyer. Bidding and selling can both be automated. Online auctions are easily scaled, too. The internet, therefore, has created a golden age for auctions.

It was predicted that small businesses would no longer need intermediaries to sell their products and services. Because the cost of reaching customers is virtually zero, it would be easy. However, teh cost of reaching people's attention suddenly became much greater, therefore revealing the need for an online market place.

\textbf{eBay} is one of the most famous auction venue online. Amazon tried to offer auctions, but eBay retained their title. The auctions on this website are open-ascending with a preset deadline. It's very similar to an English auction, but instead of waiting for a last bid, it has this set deadline. The time constraint has an affect on bidder behaviour, however. It can lead to a very intense last minute bid (sniping). The problem with sniping is that the majority of the bid length does not receive any attention, and then suddenly spike.

To combat this, eBay introduced their own `sniper' software. They made it such that you can enter a value, and will increment slowly up to the limit. Another possible fix is to extend the deadline slightly whenever a bid is placed. Amazon used this approach.

The trade off between posted prices and auctions is a trade off between price discovery and convenience. Auctions are more likely to sell, but they will sell for a lower price than a fixed price. Auctions are particularly useful for when a seller is unsure of the value of an item.

While eBay holds a lot of auctions, \textbf{Google Ads} hold the highest volume of auctions. Due to the \textit{idiosyncratic} nature of adverts, it is impossible to price an advert effectively. Therefore, Google sells them via auction. Advert quality depends on three factors:
\begin{itemize}
    \item Historic click-through-rate
    \item Relevancy
    \item Landing page quality and load speed
\end{itemize}

The highest bidder might not win the auction, since the quality of the advert also matters.

Google used to use a first-price auction (pay what you bid). Unfortunately, Google noticed that bidders would load the servers by constantly monitoring the system to work out if they could reduce their bid. Instead, they went for a second-price value approach instead, meaning that they bid their true value.

As the price set depends on the number of bidders, Google make it easier for advertisers to match (such as partial matching).

\subsubsection{Double Auction}
A reverse auction is an auction whereby the sellers offer their goods at lower and lower values.

A double auction is a combination of this and a normal auction. When the offers match, we can match the buyer and seller (think GE).

\section{Financial Technology}
There are three dimensions of financial technology:
\begin{itemize}
    \item The business functions offered
    \item Technologies and concepts
    \item Institutions in the space
\end{itemize}

This is sometimes called the \textit{finance cube}.

The traditional banking and finance firms look to use technology to automate or improve a traditional business function. They want to increase productivity and efficiency; looking to maximise profit and overlay new services. It's a new form of incremental innovation. This advancement is \textbf{digitalisation}: leveraging digital technology to improve processes.

Start-up technology companies, on hte other hand, look to introduce \textbf{new kinds} of financial services. The finance is redesigned from the ground up, making use of the technology possibilities. They want to disrupt traditional finance through new tech. This development is considered as \textbf{FinTech}.

Finally, big-tech companies already have a huge network of users. They want to streamline technology processes and lock in users. This development is considered as \textbf{TechFin}, leveraging current tech platforms to overlay financial services.

Note that these terms are \textbf{not used consistently}, which is not incorrect, but annoying for us, so we will stick to these definitions for ease of understanding.

Finance and technology have an old relationship. They have been in cahoots for a really long time. 

\subsection{Development Landscape}
According to a 2015 survey, the peak age group of financial technology users are relatively young, with the highest users being between 25 and 34. This shouldn't come as much of a surprise. The users also tend to be wealthy. However, the survey was only conducted in Australia, Canada, HK, Singapore, UK and USA. This begs the question, does this skew the result?

These countries have access to banking, but 70\% of the world's adult population are un/underbanked. This category of people need easy access, appropriate and attractive (to use) products.

Financial services have been regulated for a long time. This is to:
\begin{itemize}
    \item Protect actors
    \item Improve efficiency
    \item Reduce risk
    \item Help build trust
\end{itemize}

These \textit{financial regulators} are the reason that banks are more popular than giving your money to some random fella, and to stop them from just running away with your money. However, fintech/techfin is disruptive, so may not be initially regulated. Tech companies do not have expertise of regulatory compliance. Some regulators take a `watch and see' approach (also called sandboxing).

\subsubsection{LASIC}
\begin{itemize}
    \item \textbf{Low profit margin} --- It is necessary to be able to maintain a low profit margin. These margins remain low at the user level, so high volumes are necessary.
    \item \textbf{Asset light} --- Fixed costs should be kept low, so look to ride on existing infrastructure whenever possible
    \item \textbf{Scalable} --- Easily scalable
    \item \textbf{Innovative} --- Products and operations should be innovative, ideally a new business model
    \item \textbf{Compliance easy} --- Governments wants something that is easy to regulate.
\end{itemize}

\section{Economic Agents and Market Based systems}
Market economics is an important metaphor for new methods in computer science or engineering, dealing with allocation of scarce resources. It requires software versions of both traders and marketplaces. Designing these things is hard, but can be done automatically. 

It is useful to distinguish between \textbf{macro} and \textbf{micro}economics. Macro is the high level stuff that we don't need, and micro is just maths, which is more useful for problem solving.

There are lots of really important resource limited systems:
\begin{itemize}
    \item Telecom, computer OS, logistics, staffing, etc.
    \item Automating their control or regulation is attractive for the usual reasons
    \item and difficult for the usual reasons too
\end{itemize}

So, ideas from microeconomics could help us do automated dynamic resource allocation and control in engineered systems. Market based control is not a radical new idea, and has been around for a number of years. It happens that these automatically optimised robot traders and market mechanisms are originally intended for control of data centres and are of significant interest to major investment banks, buyside funds, and exchange operators.

\subsection{Background Economics}
From everyday experience, we're used to the idea that if the demand for some tradable item exceeds the supply, then the price will rise. This is known as a \textbf{shortage}, or \textit{seller's market}

Conversely, if the quantity supplied is greater than that demanded, the price will fall, known as a \textbf{surplus}, or \textit{buyers market}

Formally, the price that buyers are prepared to pay at each possible quantity is referred to as the demand. This, when plotted, usually slopes downward. Similarly, the price that sellers are prepared to sell for at each possible quantity is referred to as the supply, and this will slope upward on a graph.

The price will balance at an intermediate price, where the quantity supplied will equal the quantity demanded.

This is known as \textit{self-correcting}, and free markets will do this since transactions take place at pries away from equilibrium. Competition among buyers or seller will move prices back toward the equilibrium. At this equilibrium, traders have no incentive to change their prices.

Such market mechanisms can give efficient allocation of resources without a central controller or external intervention. A common ideal of efficient allocation is the notion of \textbf{Pareto efficiency}:
\begin{itemize}
    \item An allocation is Pareto efficient if no-one can be made better off without someone being made worse off.
    \item Pareto efficient allocations can arise from free markets despite the fact that each trader in the market is acting only to serve his or her self-interest
    \item The traders appear to be led by an invisible hand
    \item Free markets are not guaranteed to achieve optimal allocations. conditions in which they fail are well known (such as a monopoly).
\end{itemize}

In reality, the supply and demand curve isn't as smooth as it is in theory. This is a simplification that is really useful for teaching purposes. However, in the markets that we will be looking at, things are gonna get more complicated. It looks more \textit{stepped}. Each step in the supply curve represents an additional unit available for sale at the indicated price, while each step in the demand curve represents an additional unit desired at the indicated price. These curves also don't stand still because the buyers become saturated, or the sellers run out of stock. Even without transactions, traders dynamically alter their margins.

\subsection{Market Metrics}
\begin{itemize}
    \item \textbf{Smith's $\alpha$} --- Root mean square deviation of transaction prices around theoretical equilibrium price, expressed as a percentage (lower is better).
    \begin{align*}
    \alpha = \frac{100}{P_0} \sqrt{\sum_{t \in T} \frac{(P_t - P_0)^2}{|T|}}
    \end{align*}
    \item \textbf{Allocative Efficiency} --- Total utility (profit) earned by all traders, divided by theoretical max possible total utility (surplus), expressed as a percentage. Measures how effective the market is at extracting gains through trade.
    \item \textbf{Single Agent Efficiency} --- Profit earned by an agent, divided by its expected profit if all trades take place at the equilibrium price (how effective a single trader is)
\end{itemize}

The results of an experiment of random agents creating artificial trades concluded that most of the intelligence is in the market, and not the traders. These `zero-intelligence' agents were surprisingly human like when they were constrained to make offers and bids that would not make them a loss.

<<<<<<< HEAD
These results are actually gained, but they are \textbf{special cases}. We can actually predict when these ZI-C markets will fail to reach the equilibrium. For this reason, we can develop a better trader, known as a ZI-Plus trader.

These traders still work on their quota, as ZI-C does, but also use some machine learning; adjusting this using machine learning rules. They were able to succeed in markets where ZI-C failed.

\subsection{Maths Behind ZI-P}
ZI-C agents are hard bounded by the supply-demand curve. They are not able to offer prices that would breach this and cause them a loss. If we look at these curves, we get a probability density function (it would be equivalent to the area above the supply curve for sellers, or the area below the demand curve for buyers)

Transactions only occur when a bid and offer price cross. If we overlay the pdf for both supply and demand curve, we get the probability density function for the equilibrium price. Note that we are dealing with symmetric supply and demand curve here, which affects this slightly. The peak of this pdf is the equilibrium price. This means that it is not surprising that ZI-C agents reach equilibrium since this is the most likely value.

In asymmetric markets, the mid point (expected value) is no longer the peak. It could actually be the first point, if there is elastic supply, meaning ZI-C would fail to converge in these markets.

Very informally, the ZI-P sellers have a thing to sell, and a \textit{limit price} such that it will not lose money. The price asked for $P$ is $L$ and sme profit margin $M$: $P=(1.0+M)L$.
\begin{itemize}
    \item If we have price $P$ and:
\begin{itemize}
    \item Sellers are accepting bids below $P$ or
    \item Sellers are making offers below $P$
\end{itemize}
\item then decrease $M$ (not below zero). Otherwise:
\begin{itemize}
    \item If trades are happening at prices above $P$ then increase $M$ (even if we have nothing to sell)
\end{itemize}
\end{itemize}

Each ZI-P seller keeps their values private. The quantity in which the margin steps up or down is aiming for a target price, that is roughly the last quote price in the market. There is a little bit of randomness, too. ZI-P uses the Widrow-Hoff learning rule with \textbf{momentum (damping) factor}.

What makes a trader `best'? It is really hard to compare them, so instead of analysing them, \textit{empirical studies} are the method of choice. More on this later.

Testing agents in a \textit{homogeneous} market is sometimes not good, since for things like financial markets, this is impossible. This exact problem is what led to MGD (modified GD agent). For this reason, the following tests were performed:
\begin{itemize}
    \item \textbf{Homogeneous population test} --- Tested in a market of a single agent
    \item \textbf{One-in-many test} --- Tested in a market where all but one trader is using a different method, and only one instance of the agent
    \item \textbf{Balanced test} --- Tested in a completely split market where each agent has a counterpart
\end{itemize}
=======
>>>>>>> 00952d319c1dd6a3e0757453978a98576531d858

\end{document}