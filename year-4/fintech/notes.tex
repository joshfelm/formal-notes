% Preamble {{{
\documentclass[11pt,a4paper,titlepage,dvipsnames,cmyk]{scrartcl}
\usepackage[english]{babel}
\typearea{12}
% }}}

% Set indentation and line skip for paragraph {{{
\setlength{\parindent}{0em}
\setlength{\parskip}{1em}
\usepackage[margin=2cm]{geometry}
\addtolength{\textheight}{-1in}
\setlength{\headsep}{.5in}
% }}}

\usepackage{hhline}
\usepackage[table]{xcolor}
\usepackage{mathtools}
\usepackage[T1]{fontenc}

% Headers setup {{{
\usepackage{fancyhdr}
\pagestyle{fancy}
\lhead{Internet Economics and Financial Technology}
\rhead{Josh Felmeden}
\usepackage{hyperref}
% }}}

% Listings {{{
\usepackage[]{listings}
\lstset
{
    breaklines=true,
    tabsize=3,
    showstringspaces=false
}

\definecolor{lstgrey}{rgb}{0.05,0.05,0.05}
\usepackage{listings}
\makeatletter
\lstset{language=[Visual]Basic,
backgroundcolor=\color{lstgrey},
frame=single,
xleftmargin=0.7cm,
frame=tlbr, framesep=0.2cm, framerule=0pt,
basicstyle=\lst@ifdisplaystyle\color{white}\footnotesize\ttfamily\else\color{black}\footnotesize\ttfamily\fi,
captionpos=b,
tabsize=2,
keywordstyle=\color{Magenta}\bfseries,
identifierstyle=\color{Cyan},
stringstyle=\color{Yellow},
commentstyle=\color{Gray}\itshape
}
\makeatother
\renewcommand{\familydefault}{\sfdefault}
\newcommand{\specialcell}[2][c]{%
\begin{tabular}[#1]{@{}c@{}}#2\end{tabular}}
% }}}

% Other packages {{{
\usepackage{graphicx}
\graphicspath{ {./pics/} }
\usepackage{needspace}
\usepackage{tcolorbox}
\usepackage{soul}
\usepackage{babel,dejavu,helvet}
\usepackage{amsmath}
\usepackage{booktabs}
\usepackage{tcolorbox}
\usepackage[symbol]{footmisc}
\renewcommand{\thefootnote}{\fnsymbol{footnote}}
\renewcommand{\familydefault}{\sfdefault}
\usepackage{enumitem}
\setlist{nolistsep}
% }}}

% tcolorbox {{{
\newtcolorbox{blue}[3][] {
colframe = #2!25,
colback = #2!10,
#1,
}

\newtcolorbox{titlebox}[3][] {
colframe = #2!25,
colback = #2!10,
coltitle = #2!20!black,
title = {#3},
fonttitle=\bfseries
#1,
}
% }}}

% Title {{{
\title{Internet Economics and Financial Technology}
\author{Josh Felmeden}
% }}}

\begin{document}
\maketitle
\tableofcontents
\newpage
\section{The Big Picture}
Based on past events, it is possible that we have reached the climax of the IT revolution. The migration to cloud technology is mirrored by the migration of electricity from home generators to power stations. From here, there have only been marginal improvements in this field, and this could be the same in the tech sector.

Similarly, this was seen in the explosion of the `.com boom' seen later. This rapid growth was due to people investing lots of money in the technology; a mania. Again, this has been seen before in history with the development of canals.

Another bubble popping phenomena was the collapse of the American housing market, where the stock market collapsed. Humans are really good at messing up financially, and also inventing tech that can revolutionise the world.

Essentially, what I'm getting at is that in the financial sector, trading is mostly done by computers. Obviously, these computer traders have no common sense, which worries a lot of the big companies, and investigations have been done into this.

\subsection{Expanding the big picture}
This section is the only section where we will discuss the history of finance, so bear with me here. Looking at the last 250 years of technology `surges', we can possible gain insights about the current surge.
\begin{enumerate}
\item Industrial revolution (1770-1873)
\item Steam and railways (1829-1873)
\item Steel, electricity (1875-1918)
\item Oil, Car, Mass production (1908-1974)
\item IT and Telecoms (1971-??)
\end{enumerate}

If the IT surge ends at the mean duration, then we can expect the surge to end in 2024. This end does not mean the end of the world, just that the time to make great fortunes has passed.

So, computer science is no longer \textit{just} about computer science. Because of this surge, there is great interplay between this field and others, such as finance.

\section{Big Money}
\subsection{Positive Feedback}
Successful companies have won out for a whole host of factors, but history tends to focus on the winners. \textit{Positive feedback} and \textit{network externalities} can help to elevate a company or product to success, even if it has superior rivals.

\subsection{The Long Tail}
The long tail was named and popularised by Chris Anderson. In the old days, Retailers could make money on high-volume, low-margin goods or low-volume, high-margin goods. This is because shops had shelf volume that, if filled up by low-volume, low-margin goods, would be making a loss on these products, since physical space costs money. Thanks to the advent of online markets, `shelf' space costs almost nothing, meaning that shops can now stock the low-volume, low-margin goods.

In examples such as the music industry, these less popular goods still have returns that would be beneficial for the vendor. In fact, no matter how far down the popularity graph you go, there is still money to be made. This tail is called the \textit{long tail}.

\subsubsection{Power-Law distribution of popularity}
The power law appears a weirdly large amount in terms of popularity, which is of the form $P = cR^{-v}$. This makes the graph dip very sharply early on, but level off; never reaching zero.

As it turns out, the biggest money could be made in the smallest scales (if they are done large scale). The way to do it is:
\begin{enumerate}
\item Make everything available
\item Reduce prices
\item `Help me find it'
\end{enumerate}

\subsubsection{Criticism of The Long Tail}
Does the long-tail effect really exist? One observation is that the web actually \textit{magnifies} the importance of `blockbuster' hits. However, Anderson disagrees with this, stating that it is all about where you define the head and tail of the power-law curve and whether you use absolute values (since the criticism compares percentages).

Another criticism stated that music sales exhibited a log-normal distribution rather than a power-law curve. They reported that 80\% of the music tracks they monitored sold NO copies at all over a one-year period. This was disputed poorly by Anderson.

\subsubsection{Support for The Long Tail}
The Long Tail has been proven to have grown longer over time. Niche books now account for around 37\% of Amazon's sales. Additionally, a longer but also fatter tail has been observed on consumer software downloading patterns. However, this is only for software, which would behave differently than the entertainment sector.

\subsection{Disruptive Technology}
A technology company may have some form of technology and a projected performance. The company will also have an idea of the improvement required by the mainstream market. As long as the technology offered improves faster than the mainstream market's demands, the company is going to do fine. However, these are just predictions, and therefore this prediction must be monitored.

Some time into the future, a new technology appears with a much smaller performance and market than yours. However, it may have another benefit (smaller/cheaper/lighter), meaning that other companies invest in this. Because our company is already doing well, it is not a threat and we are not interested in it. The research team also conclude that the technology will not improve as fast as ours, therefore, our technology will always outperform the new one.

This prediction is true for a long time. At a certain point, the needs of the mainstream market are met by the new technology. Despite our technology being far above the needs of the consumer, the market is taken away from us, because our technology is far too advanced, and the new technology has other benefits that the mainstream market find more attractive, causing our company to be obsolete.

\subsubsection{The Innovator's Dilemma}
There are two different types of innovations:
\begin{itemize}
    \item \textbf{Sustaining Innovations}: Incremental improvements on existing products or services that are attractive to existing customers and business models. Eventually, you offer more than the customer wants
    \item \textbf{Disruptive Innovations}: perform perform less well than existing products, perhaps being lower quality or less sophisticated, but they are also simpler, cheaper, or more user friendly.
\end{itemize}

These disruptive innovations can cause strong incumbent companies to fall or falter, not due to weaknesses in these companies, but because they do the right think short-term; leaving the lower end of the market to others.

The \textit{dilemma} is the choice between doing what made the company a success in the first place, or investing in a lower quality prospect, meaning that sometimes successful incumbents need to invest in the `wrong thing'.

Very often, this is the sequence of events:
\begin{itemize}
    \item The disruptive technology is developed by the incumbent company
    \item The existing customers are unimpressed
    \item Therefore, the company stops developing the disruptive technology and instead concentrates on sustaining innovations
    \item New companies form (sometimes from disgruntled ex-employees of said company) and develop a new market for the disruptive technology
    \item As the disruptive technology matures and improves, it moves `up the chain'
    \item The company then realises that there is significant demand, and attempts to enter as a latecomer. It fails to do this due to the lead built already by these new companies.
\end{itemize}
\end{document}