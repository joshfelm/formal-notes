% Preamble {{{
\documentclass[11pt,a4paper,titlepage,dvipsnames,cmyk]{scrartcl}
\usepackage[english]{babel}
\typearea{12}
% }}}

% Set indentation and line skip for paragraph {{{
\setlength{\parindent}{0em}
\setlength{\parskip}{1em}
\usepackage[margin=2cm]{geometry}
\addtolength{\textheight}{-1in}
\setlength{\headsep}{.5in}
% }}}

\usepackage{hhline}
\usepackage[table]{xcolor}
\usepackage{mathtools}
\usepackage[T1]{fontenc}

% Headers setup {{{
\usepackage{fancyhdr}
\pagestyle{fancy}
\lhead{Internet Economics and Financial Technology}
\rhead{Josh Felmeden}
\usepackage{hyperref}
% }}}

% Listings {{{
\usepackage[]{listings}
\lstset
{
    breaklines=true,
    tabsize=3,
    showstringspaces=false
}

\definecolor{lstgrey}{rgb}{0.05,0.05,0.05}
\usepackage{listings}
\makeatletter
\lstset{language=[Visual]Basic,
backgroundcolor=\color{lstgrey},
frame=single,
xleftmargin=0.7cm,
frame=tlbr, framesep=0.2cm, framerule=0pt,
basicstyle=\lst@ifdisplaystyle\color{white}\footnotesize\ttfamily\else\color{black}\footnotesize\ttfamily\fi,
captionpos=b,
tabsize=2,
keywordstyle=\color{Magenta}\bfseries,
identifierstyle=\color{Cyan},
stringstyle=\color{Yellow},
commentstyle=\color{Gray}\itshape
}
\makeatother
\renewcommand{\familydefault}{\sfdefault}
\newcommand{\specialcell}[2][c]{%
\begin{tabular}[#1]{@{}c@{}}#2\end{tabular}}
% }}}

% Other packages {{{
\usepackage{graphicx}
\graphicspath{ {./pics/} }
\usepackage{needspace}
\usepackage{tcolorbox}
\usepackage{soul}
\usepackage{babel,dejavu,helvet}
\usepackage{amsmath}
\usepackage{booktabs}
\usepackage{tcolorbox}
\usepackage[symbol]{footmisc}
\renewcommand{\thefootnote}{\fnsymbol{footnote}}
\renewcommand{\familydefault}{\sfdefault}
\usepackage{enumitem}
\setlist{nolistsep}
% }}}

% tcolorbox {{{
\newtcolorbox{blue}[3][] {
colframe = #2!25,
colback = #2!10,
#1,
}

\newtcolorbox{titlebox}[3][] {
colframe = #2!25,
colback = #2!10,
coltitle = #2!20!black,
title = {#3},
fonttitle=\bfseries
#1,
}
% }}}

% Title {{{
\title{Internet Economics and Financial Technology}
\author{Josh Felmeden}
% }}}

\begin{document}
\maketitle
\tableofcontents
\newpage
\section{The Big Picture}
Based on past events, it is possible that we have reached the climax of the IT revolution. The migration to cloud technology is mirrored by the migration of electricity from home generators to power stations. From here, there have only been marginal improvements in this field, and this could be the same in the tech sector.

Similarly, this was seen in the explosion of the `.com boom' seen later. This rapid growth was due to people investing lots of money in the technology; a mania. Again, this has been seen before in history with the development of canals.

Another bubble popping phenomena was the collapse of the American housing market, where the stock market collapsed. Humans are really good at messing up financially, and also inventing tech that can revolutionise the world.

Essentially, what I'm getting at is that in the financial sector, trading is mostly done by computers. Obviously, these computer traders have no common sense, which worries a lot of the big companies, and investigations have been done into this.

\subsection{Expanding the big picture}
This section is the only section where we will discuss the history of finance, so bear with me here. Looking at the last 250 years of technology `surges', we can possible gain insights about the current surge.
\begin{enumerate}
\item Industrial revolution (1770-1873)
\item Steam and railways (1829-1873)
\item Steel, electricity (1875-1918)
\item Oil, Car, Mass production (1908-1974)
\item IT and Telecoms (1971-??)
\end{enumerate}

If the IT surge ends at the mean duration, then we can expect the surge to end in 2024. This end does not mean the end of the world, just that the time to make great fortunes has passed.

So, computer science is no longer \textit{just} about computer science. Because of this surge, there is great interplay between this field and others, such as finance.
\end{document}