% Preamble {{{
\documentclass[11pt,a4paper,titlepage,dvipsnames,cmyk]{scrartcl}
\usepackage[english]{babel}
\typearea{12}
% }}}

% Set indentation and line skip for paragraph {{{
\setlength{\parindent}{0em}
\setlength{\parskip}{1em}
\usepackage[margin=2cm]{geometry}
\addtolength{\textheight}{-1in}
\setlength{\headsep}{.5in}
% }}}

\usepackage{hhline}
\usepackage[table]{xcolor}
\usepackage{mathtools}
\usepackage[T1]{fontenc}

% Headers setup {{{
\usepackage{fancyhdr}
\pagestyle{fancy}
\lhead{Advanced Topics with Programming Languages}
\rhead{Josh Felmeden}
\usepackage{hyperref}
% }}}

% Listings {{{
\usepackage[]{listings}
\lstset
{
    breaklines=true,
    tabsize=3,
    showstringspaces=false
}

\definecolor{lstgrey}{rgb}{0.05,0.05,0.05}
\usepackage{listings}
\makeatletter
\lstset{language=[Visual]Basic,
backgroundcolor=\color{lstgrey},
frame=single,
xleftmargin=0.7cm,
frame=tlbr, framesep=0.2cm, framerule=0pt,
basicstyle=\lst@ifdisplaystyle\color{white}\footnotesize\ttfamily\else\color{black}\footnotesize\ttfamily\fi,
captionpos=b,
tabsize=2,
keywordstyle=\color{Magenta}\bfseries,
identifierstyle=\color{Cyan},
stringstyle=\color{Yellow},
commentstyle=\color{Gray}\itshape
}
\makeatother
\renewcommand{\familydefault}{\sfdefault}
\newcommand{\specialcell}[2][c]{%
\begin{tabular}[#1]{@{}c@{}}#2\end{tabular}}
% }}}

% Other packages {{{
\usepackage{graphicx}
\graphicspath{ {./pics/} }
\usepackage{needspace}
\usepackage{tcolorbox}
\usepackage{soul}
\usepackage{babel,dejavu,helvet}
\usepackage{amsmath}
\usepackage{booktabs}
\usepackage{tcolorbox}
\usepackage[symbol]{footmisc}
\renewcommand{\thefootnote}{\fnsymbol{footnote}}
\renewcommand{\familydefault}{\sfdefault}
\usepackage{enumitem}
\setlist{nolistsep}
% }}}

% tcolorbox {{{
\newtcolorbox{blue}[3][] {
colframe = #2!25,
colback = #2!10,
#1,
}

\newtcolorbox{titlebox}[3][] {
colframe = #2!25,
colback = #2!10,
coltitle = #2!20!black,
title = {#3},
fonttitle=\bfseries
#1,
}
% }}}

% Title {{{
\title{Advanced Topics with Programming Languages}
\author{Josh Felmeden}
% }}}

\begin{document}
\maketitle
\tableofcontents
\newpage

\section{Introduction}
One way of looking at programming languages is to look at \textbf{types} and \textbf{type systems}. Haskell is a language that uses typing. There can be static and dynamic typing.

%definition
Types classify programs by the kind of data they compute.

\section{Judgements}
A \textbf{judgement} is a statement. In this topic, we will centre everything around an \textit{evident judgement}. A judgement becomes evident when you can \textit{prove} it. Therefore, when we sa a judgement, we need to provide evidence of proof.

Judgements come with rules. Here is an axiom:
\begin{lstlisting}
zero nat
\end{lstlisting}

Zero is the object, and nat is the name. Alongside this, we can use an inference rule:
\begin{lstlisting}
n nat           'premise(s)
------------- s 'name of rule
succ(n) nat     'conclusion
\end{lstlisting}

These two structures can be used in \textbf{derivation trees} which are used to prove judgements. For example, to prove that two is a natural number, we can do the following:
\begin{lstlisting}

--------- axiom
zero nat
-------- s
succ(zero) nat
--------- s
succ(succ(zero)) nat
\end{lstlisting}

We can also write
\begin{lstlisting}
data nat = zero | succ nat
\end{lstlisting}

\subsection{Simultaneous rules}
we can state proofs of rules mix and match to use a proof that proves two things at once. For example:
\begin{lstlisting}

-------- ZE
zero even
\end{lstlisting}

\begin{lstlisting}
n even
---------- ODD
succ(n) odd
\end{lstlisting}

\begin{lstlisting}
n odd
----------- EVEN
succ(n) even
\end{lstlisting}

This proves both odd and even.

\section{Induction}
Every set of rules generates an \textit{induction principle}. 

Consider the claim \lstinline|if succ(n) nat then n nat|. This seems obvious, but we can actually prove this.

\begin{blue}{blue}
\texttt{Proof} We will use induction \\
\texttt{P(n)}: `If n nat and n = succ(x) for some x then x nat' \\
\texttt{Case zero}: Nothing to prove \\
\texttt{Case(succ(n) nat)} The derivation of succ(n) nat ends with
\begin{lstlisting}
n nat
--------- succ
succ(n) nat
\end{lstlisting}
The D is a derivation of n nat.

succ(n) = succ(x) and therefore n = x. We can conclude that n is nat and therefore x is nat.
\end{blue}

This statement is an \textbf{admissible rule}. A rule is admissible when we have a derivation of the premises, then we know we can construct a derivation of the conclusion. In essence, you need to \textit{prove} this one (usually by induction).

In contrast, a rule is \textbf{derivable} if we can use a derivation of its premise as a building block in deriving its conclusion. In essence, you can \textit{infer} this one (stitch together stuff).

\subsection{Simultaneous induction}
Recalling the even and odd proof, we can write these as Let P(n even) and Q(n odd). If:
\begin{itemize}
    \item P(zero) and
    \item whenever $n$ even and $\mathcal{P}(n)$ we have $\mathcal{Q}(\text{succ}(n))$ and
    \item whenever $n$ odd and $\mathcal{Q}(n)$ we have $\mathcal{P}(\text{succ}(n))$
\end{itemize}

We are allowed to \textit{invert} a judgement, and this is called an \textit{inversion principle}.

\section{Types}
Term $e$ is \textbf{well-typed} iff there is $\tau$ such that $\emptyset \vdash e : \tau$ is derivable according the the \textit{static} rules of the language.

Say we want to prove the following:
\begin{align*}
\emptyset \vdash \text{ let}(\text{str} [ \text{my} ]; x, (\text{times}(\text{len}(x); \text{num}(0))))
\end{align*}

Type systems restrict the set of allowed programs.

\subsection{Basic properties of typing}
\textbf{Lemma (Inversion of Typing)}: Suppose that $\Gamma \vdash e : \tau$. If $e = \text{ plus}(e_1 ;e_2)$ then $\tau = \text{ num}, \Gamma \vdash e_1 : \text{ num}$ and $\Gamma \vdash e_2 : \text{ num}$ and similarly for the other constructs of the language.

\textbf{Lemma (Unicity of Typing)}: For every typing context $\Gamma$ and expression $e$ there exists at most one $\tau$ such that $\Gamma$ such that $\Gamma \vdash e : \tau$.

\textbf{Lemma (Weakening)}: If $\Gamma \vdash e' : \tau'$ then $\Gamma, x : \tau \vdash e' : \tau'$ for any $x \not \in dom(\Gamma)$ and any type $\tau$.

\textbf{Lemma (Substitution)}: If $\Gamma, x : \tau \vdash e' : \tau'$ and $\Gamma \vdash e : \tau$ then $\Gamma \vdash e'[e/x] : \tau '$

\end{document}