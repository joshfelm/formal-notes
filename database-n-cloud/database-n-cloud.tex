%%=====================================================================================
%%
%%       Filename:  database-n-cloud.tex
%%
%%    Description:  Notes for database and cloud written up nicely
%%
%%        Version:  1.0
%%        Created:  06/04/19
%%       Revision:  none
%%
%%         Author:  Josh Felmeden (), nk18044@bristol.ac.uk
%%   Organization:  
%%      Copyright:  Copyright (c) 2019, Josh Felmeden
%%
%%          Notes:  
%%
%%=====================================================================================

% Preamble
\documentclass[11pt,a4paper,titlepage,dvipsnames,cmyk]{scrartcl}
\usepackage[english]{babel}
\typearea{12}

% Set indentation and line skip for paragraph
\setlength{\parindent}{0em}
\setlength{\parskip}{1em}
\usepackage[margin=2cm]{geometry}
\addtolength{\textheight}{-1in}
\setlength{\headsep}{.5in}

% Headers setup
\usepackage{fancyhdr}
\pagestyle{fancy}
\lhead{Databases and the Cloud: The notes}
\rhead{Josh Felmeden}
\usepackage{hyperref} 
\usepackage{mathtools} 


% Listings
\usepackage[]{listings,xcolor} 
\lstset
{
    breaklines=true,
    tabsize=3,
    showstringspaces=false
}


\lstdefinestyle{Common}
{
    extendedchars=\true,
    language={[Visual]Basic},
    frame=single,
    %===========================================================
    framesep=3pt,%expand outward.
    framerule=1.4pt,%expand outward.
    xleftmargin=3.4pt,%make the frame fits in the text area. 
    xrightmargin=3.4pt,%make the frame fits in the text area.
    %=========================================================== 
    rulecolor=\color{Red}
}

\lstdefinestyle{A}
{
    style=Common,
    backgroundcolor=\color{Yellow!10},
    basicstyle=\small\color{Black}\ttfamily,
    keywordstyle=\color{Orange},
    identifierstyle=\color{Cyan},
    stringstyle=\color{Red},
    commentstyle=\color{Green}
}

\lstdefinestyle{B}
{
    style=Common,
    backgroundcolor=\color{Black},
    basicstyle=\scriptsize\color{White}\ttfamily,
    keywordstyle=\color{Orange},
    identifierstyle=\color{Cyan},
    stringstyle=\color{Red},
    commentstyle=\color{Green}
}

\usepackage[]{amsmath} 
\usepackage[]{booktabs} 
\usepackage[symbol]{footmisc} 
\renewcommand{\thefootnote}{\fnsymbol{footnote}}

% Title
\title{Databases and the Cloud: The Notes}
\date{2018\\ December}
\author{Josh Felmeden}

% Start document
\begin{document}
\pagenumbering{roman}
\maketitle

\tableofcontents
\newpage
\pagenumbering{arabic}

\section{The Internet}%
\label{sec:The Internet}
End systems are connected via the \textbf{communication links} that
consist of the different types of physical media. Usually, the end systems
are not directly attached by a single link, but rather they are attached
through a router.

There are two kinds of host: \textit{clients} and \textit{servers}. A
program or machine that responds to request and others is called a
\textbf{server} while a program or machine that sends the requests to the
server is called a \textbf{client}.

The internet is made possible by the development, testing, and
implementation of the \textit{internet standards}. They are developed by
the Internet Engineering Task Force (or the IETF). Their documents are
known as RFCs (request for comments). There are a number of protocols,
such as TCP, IP, HTTP, and SMTP (this one is used for emails). There are
more than 2000 RFCs.

\subsection{Protocols}%
\label{sub:Protocols}
A \textbf{protocol} is a set of rules that govern the communication to
ensure a standard of communication. It also consists of messages sent and
actions taken in response to replies or other such events.

A simple protocol could be where one machine sends a message (called a
\textit{request}) and another machine replies with a response. This can
then be repeated.

\subsection{Internet Layers}%
\label{sub:internet-layers}
\begin{itemize}
    \item HTTP
    \begin{itemize}
        \item Makes request
        \item Reads and handles the response
    \end{itemize}
    \item TCP
        \begin{itemize}
            \item Breaks data up into packets
            \item Puts the packets back in order and reassembles messages
        \end{itemize}
    \item IP
        \begin{itemize}
            \item Attaches to and from addresses to each packet
            \item Reads and groups packets based on the address
        \end{itemize}
    \item Physical internet
    \begin{itemize}
        \item Send bits to local routers
        \item Receives bits and assembles into packets
    \end{itemize}
\end{itemize}

\subsection{HTTP: Hyper text transfer protocol}%
\label{sub:http}
What's the difference between the web and the internet? Well, the internet
is the computer network itself (or the whole infrastructure): while the
web (or the world wide web) is an application that runs on that
infrastructure.

It's probably the most common application protocol that there is on the
web (but there are others like video streaming and FTP and the like).

\end{Document}
